\documentclass[]{article}

%%%%%%%%%%%%%%%%%%%
% Packages/Macros %
%%%%%%%%%%%%%%%%%%%
\usepackage{amssymb,latexsym,amsmath}     % Standard packages
%%%%%%%%%%%
% Margins %
%%%%%%%%%%%
\addtolength{\textwidth}{1.0in}
\addtolength{\textheight}{1.00in}
\addtolength{\evensidemargin}{-0.75in}
\addtolength{\oddsidemargin}{-0.75in}
\addtolength{\topmargin}{-.50in}

%%%%%%%%%%%%%%%%%%%%%%%%%%%%%%
% Theorem/Proof Environments %
%%%%%%%%%%%%%%%%%%%%%%%%%%%%%%
\newtheorem{theorem}{Theorem}
\newenvironment{proof}{\noindent{\bf Proof:}}{$\hfill \Box$ \vspace{10pt}}  


%%%%%%%%%%%%
% Document %
%%%%%%%%%%%%
\begin{document}

\title{ELO-330: Programación de Sistemas\\
\textbf{Tarea 2: Procesando audio usando código libre}}
\author{Camilo Barra R}
\date{}
\maketitle

\section{Introducción}
Una de las grandes ventajas de programar usando pipes es el hecho de que se
pueden utilizar aplicaciones externas al programa, es decir, como programador
no es necesario rediseñar programas, que ya realizan de forma eficiente una tarea,
más bien como programar, se deben traducir las instrucciones para poder comunicarle a programa externo lo que se desea y captar la salida si así se lo quiere.

\section{Descripción del programa}
El programa permite poder recuperar audio en formato PCM cuando la señal
se satura. Para apreciar la saturación y la ”recuperación” de la señal (usando interpolación de polinomios) se gráfica el audio original, el saturado y el recuperado, además se reproducen las señales ya mencionadas.

\section{Requerimientos para Ubuntu}
Para poder ejecutar la inaplicación se deben instalar octave (suministra la interpolación y los gráficos) y aplay (reproduce el audio) mediante las siguientes lineas
de comandos:
\begin{itemize}
\item \textbf{sudo apt-get install aplay}
\item \textbf{sudo apt-get install octave}
\end{itemize}

\section{Ejecución}
Para poder ejecutar el programa se debe realizar la siguiente instrucción:
\begin{center}
\textbf{csa audio ganancia offset p}
\end{center}
• audio: es la pista de audio en formato PCM, little endian, con signo y de
16 bits con frecuencia de muestreo debe ser de 8[KHz].
• ganancia: representa la saturación  que afecta al audio (debe ser mayor
que 0).
• offset: representa es desplazamiento en el gráfico que se desea (se considera
en múltiplos de 0.125[ms])
• p: indica si se desea reproducir los 3 audios.

\section{Problema}
Existe un error en la ejecución del programa, cuando se llama a \textit{interpolation} (linea 180), dicha función funciona correctamente para n bytes, pero al usar el tamaño completo del audio al ejecutar lanza un error de desbordamiento de buffers. 
Para poder apreciar que el programa realizar los gráficos junto con la reproducción del audio comentar la linea 180 , generando que no se intente interpolar polinomios para recuperar el audio saturado.
\end{document}